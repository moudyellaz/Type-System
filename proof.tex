\documentclass[10pt]{article}
\usepackage{framed}
\usepackage{epsfig}
\usepackage{calc}
\usepackage{amstext}
\usepackage{amsthm}
\usepackage{multicol}
\usepackage{pslatex}
\usepackage{apalike}
\usepackage{cite}
\usepackage{amsmath,amssymb,amsfonts}
\usepackage{algorithmic}
\usepackage{graphicx}
\usepackage{textcomp}
\usepackage{xcolor}
\usepackage{msc}
\usepackage{mathpartir}
\usepackage{tabularx}
\usepackage{tikz}
\usepackage{pstricks}
%\usepackage{auto-pst-pdf}
\usepackage{calc}
\usepackage{ifthen}
\setmsckeyword{}
\usepackage{float}
\usetikzlibrary{positioning}
\newcommand{\If}{\sf if}
\newcommand{\Iff}{\sf iff}
\newcommand{\Then}{\sf then}
\newcommand{\While}{\sf while}
\newcommand{\Do}{\sf do}
\newcommand{\Else}{\sf else}
\newcommand{\Branch}{\sf branch}
\newcommand{\Loop}{\sf loop}
\newcommand{\For}{\sf for}
\newcommand{\Sbc}{\sf sbroadcast}
\newcommand{\SBC}{\sf Secure \; broadcast}
\newcommand{\Ra}{\sf Ra}
\newcommand{\ra}{\sf remote \; attestation}
\newcommand{\RA}{\sf Endorse-Ra}
\newcommand{\BEnc}{\sf BEnc}
\newcommand{\CheckT}{\sf check}
\newcommand{\Ass}{\sf assign}
\newcommand{\Seq}{\sf sequence}
\newcommand{\pc}{\sf pc}
\newcommand{\Filter}{\sf filter}
\newcommand{\Enc}{\sf Enc}
\newcommand{\KG}{\sf KG}
\newcommand{\Skip}{\sf skip}
\newcommand{\Var}{\sf var}
\newcommand{\pk}{\sf pk}
\newcommand{\sk}{\sf sk}
\newcommand{\LVar}{\sf Lvar}
\newcommand{\KVar}{\sf Kvar}
\newcommand{\CVar}{\sf Cvar}
\newcommand{\AssC}{\sf assign-counter}
\newcommand{\Base}{\sf base}
\newcommand{\Cmd}{\sf cmd}
\newcommand{\Endorse}{\sf endorse}
\newcommand{\Sub}{\sf subtype}
\newtheorem{definition}{Definition}
\newtheorem{example}{Example}
\newtheorem{theorem}{Theorem}
\newtheorem{lemma}{Lemma}
\newtheorem{case}{Case} 
\newtheorem{subcase}{Subcase} 
\newtheorem{subsubcase}{Subcase}
\numberwithin{subcase}{case}
\numberwithin{subsubcase}{subcase}
\begin{document}
%Theorem: if $\Gamma \vdash p$ then $p$ is $NI^{\Gamma}_\tau$. We prove this theorem by induction on the height of typing derivation tree $\Gamma \vdash p$.

\begin{definition}[$\tau-Equal$ Memories]\label{def1}
Two memories $\mu_0$, $\mu_1$ are $\tau-Equal$ for $\Gamma$, written $\mu_0 =_{\tau}^{\Gamma} \mu_1$,  $\Iff$  
$dom(\mu_0) = dom(\mu_1)$ $\land$ $\forall x \in \mu_0$ \textit{such that} $\If \; \Gamma(x) = \tau' \; var \wedge \tau' \leq \tau$, then $\mu_0(x) = \mu_1(x)$.
\end{definition}

\begin{definition}[Filtering]\label{def2} Let $\Filter_{\tau}(t) = \Filter'_{\tau}(t, \,[ \; ]\,)$ and $\Filter'_{\tau}(\,[ \; ]\,, t) = t$. 
\end{definition}
\begin{equation*}
\Filter'_{\tau}(\BEnc(L, vk, v'||v) \cdot t', t) = \\
\begin{cases} 
\Filter'_{\tau}(t', t \cdot (L, v')) & \If \; \Gamma(L) \leq \tau \\ 
\Filter'_{\tau}(t', t) & Otherwise \end{cases} 
\end{equation*}

\begin{definition}[NonInterference]\label{def3} A server program $p$ is $NI$ at $\tau$ for $\Gamma$, written $NI_{\tau}^{\Gamma}(p)$, $\Iff$ 
 $\forall \mu_0, \mu_1$ \textit{such that} $\mu_0 =_{\tau}^{\Gamma} \mu_1$ $\land$
$\mu_0 \vdash p \Rightarrow^{t_0} \mu'_0$ $\land$ 
$\mu_1 \vdash p \Rightarrow^{t_1} \mu'_1$, then $\mu'_0 =_{\tau}^{\Gamma} \mu'_1 \wedge \Filter(t_0) = \Filter(t_1)$. \\
\end{definition}


\begin{lemma}(LowExpression)\label{lemma1}
$\forall \tau, \tau', \mu_0, \mu_1$, if $\mu_0 =_{\tau}^{\Gamma} \mu_1$ and $\Gamma \vdash e: \tau'$ and $\tau' \leq \tau$ and $\mu_0 \vdash e \Rightarrow v_0$ and $\mu_1 \vdash e \Rightarrow v_1$, then $v_0 = v_1$.
\end{lemma}

%\begin{lemma}(HighExpression)\label{lemma2}
%For any two $\tau-Equal$ memories$\mu_0 =_{\tau}^{\Gamma} \mu_1$, there exist two memories $\mu'_0 =_{\tau}^{\Gamma} \mu'_1$ that agree with $\mu_0$ and $\mu_1$ except for brackets. We put brackets around the values of all high variables. Conversely, for any  $\mu'_0 =_{\tau}^{\Gamma} \mu'_1$, where brackets are only for high variables, there are two $\tau-Equal$ memories $\mu_0 =_{\tau}^{\Gamma} \mu_1$.
%\end{lemma}


\begin{theorem}\label{theorem1}
If $\Gamma \vdash p$ then $p$ is $NI^{\Gamma}_\tau$.
\end{theorem}
\begin{proof}

By Definition~\ref{def3}, for $p$ to be $NI^{\Gamma}_\tau$, we need to prove that:
\begin{itemize}
\item[($G_1$)] $\mu'_0 =_{\tau}^{\Gamma} \mu'_1$.
\item[($G_2$)] $\Filter_{\tau}(t_0) = \Filter_{\tau}(t_1)$.
\end{itemize}
We prove this theorem by induction on the height of typing derivation tree $\Gamma \vdash p$.

\case{Base case: height = 2}

\subcase{$p \overset{\Delta}{=} x:=e'$.}

Concerning ($G_1$) and ($G_2$), by Definition~\ref{def3}, we have that:
\begin{itemize}
\item[ ($H_1$)] $\mu_0 =_{\tau}^{\Gamma} \mu_1$.
\item[($H_2$)] $\mu_i \vdash x:=e' \Rightarrow^{t_i} \mu'_i$.
\end{itemize}

Moreover, for the semantics rule Update we have that:
\begin{itemize}
\item[($H_3$)]  $t_i = \epsilon$.
\end{itemize}

Hence, by the semantics rule Update and ($H_2$), we have that:
\begin{itemize}
\item[ ($H_4$)] $\mu_i \vdash e' \Rightarrow v_i$.
\item[($H_5$)] $\mu'_i = \mu_i[x:=v_i]$.
\end{itemize}

By the hypothesis of Theorem~\ref{theorem1}, we have:
\begin{itemize}
\item[($H_6$)] $\Gamma \vdash x:= e': \tau' \; cmd$.
\end{itemize}
 
By ($H_6$) and the typing rule Assign, we have that:
\begin{itemize}
\item[($H_7$)] $\Gamma \vdash x: \tau' \; var$.
\item[($H_8$)] $\Gamma \vdash e': \tau'$.
\end{itemize}

 By ($H_7$) and the typing rule Var, we have:
 \begin{itemize}
 \item[($H_9$)] $\Gamma(x) = \tau' \; var$.
 \end{itemize}
 
 


%\begin{lemma}(LowExpression)\label{lemma1}
%$\forall \tau, \tau', \mu_0, \mu_1$, if $\mu_0 =_{\tau}^{\Gamma} \; \& \; \Gamma \vdash e: \tau' \; \& \; \tau' \leq \tau \; \& \; \mu_0 \vdash e \Rightarrow v_0 \; \& \; \mu_1 \vdash e \Rightarrow v_1$ then $v_0 = v_1$.
%\end{lemma}

Depending on $\tau'$, we have two cases: 
\begin{itemize}
\item[$(H_{10})$] $\tau' \leq \tau$.

By Lemma~\ref{lemma1} that can be applied due to $(H_1)$, $(H_8)$, $(H_{10})$ and $(H_4)$ then \\
$(H_{11}): \; v_0 = v_1$. \\
To prove ($G_1$), we rely on the Definition~\ref{def1} that states that $\mu_0 =_{\tau}^{\Gamma} \mu_1$, if $\forall x \in \mu_0$ such that $\Gamma(x) = \tau' \; var$ and $\tau' \leq \tau$ then $\mu_0(x) = \mu_1(x)$.  \\
Since $(H_1)$ holds and the only variable in which $\mu_i$ and $\mu'_i$ are different is $x$ by $(H_5)$, then we need to prove that $\mu_0(x) = \mu_1(x)$ which holds by $(H_{11})$. \\
To prove ($G_2$) we rely on the Definition~\ref{def2} and $(H_3)$. Since $t_0 = t_1 = \epsilon$ then $\Filter_{\tau}(t_0) = \Filter_{\tau}(t_1)$.

Since we proved ($G_1$) and ($G_2$), then $p$ is $NI^{\Gamma}_\tau$.

\item[$(H_{12})$] $\tau' \not\leq \tau$.

To prove ($G_1$), we rely on the Definition~\ref{def1} that states that $\mu_0 =_{\tau}^{\Gamma} \mu_1$, if $\forall y \in \mu_0$ such that $\Gamma(y) = \tau' \; var$ and $\tau' \leq \tau$ then $\mu_0(s) = \mu_1(y)$. For $(H_{12})$, we have that $\tau' \not\leq \tau$. Since we are only interested by variables with security level less or equal than $\tau$, we can conclude by $(H_5)$ and $(H_1)$ that $\mu'_0 =_{\tau}^{\Gamma} \mu'_1$.

To prove ($G_2$) we rely on the Definition~\ref{def2} and $(H_3)$. Since $t_0 = t_1 = \epsilon$ then $\Filter_{\tau}(t_0) = \Filter_{\tau}(t_1)$.

Since we proved ($G_1$) and ($G_2$), then $p$ is $NI^{\Gamma}_\tau$.

%To prove ($G_1$) we rely on  Lemma~\ref{lemma2}, that states that for any two $\tau-Equal$ memories $\mu_0 =_{\tau}^{\Gamma} \mu_1$, there exist two memories $\mu'_0 =_{\tau}^{\Gamma} \mu'_1$ that agree with $\mu_0$ and $\mu_1$ except for brackets. We let $x$ be a high variable. For ($H_5$) we have that $\mu'_i = \mu_i[x:=v_i]$. Since ($H_1$) holds and by Lemma~\ref{lemma2}, we have that $\mu'_i = \mu_i$. Since $\mu_0 =_{\tau}^{\Gamma} \mu_1$ and $\mu'_0 = \mu_0$ and  $\mu'_1 = \mu_1$, then $\mu'_0 =_{\tau}^{\Gamma} \mu'_1$. 
%
%To prove ($G_2$) we rely on the Definition~\ref{def2} and $(H_3)$. Since $t_0 = t_1 = \epsilon$ then $\Filter_{\tau}(t_0) = \Filter_{\tau}(t_1)$.
%
%Since we proved ($G_1$) and ($G_2$), then $p$ is $NI^{\Gamma}_\tau$.
\end{itemize}


\subcase{$p \overset{\Delta}{=} \pc:=\pc +1 $.}


Concerning ($G_1$) and ($G_2$), by Definition~\ref{def3}, we have that:
\begin{itemize}
\item[ ($H_1$)] $\mu_0 =_{\tau}^{\Gamma} \mu_1$.
\item[($H_2$)] $\mu_i \vdash \pc:=\pc + 1 \Rightarrow^{t_i} \mu'_i$.
\end{itemize}


By the semantics rule Update we have that:
\begin{itemize}
\item[($H_3$)]  $t_i = \epsilon$.
\end{itemize}

Moreover, by the semantics rule Update and ($H_2$), we have that:
\begin{itemize}
\item[ ($H_4$)] $\mu_i \vdash \pc + 1 \Rightarrow v_i$.
\item[($H_5$)] $\mu'_i = \mu_i[x:=v_i]$.
\end{itemize}

By the hypothesis of Theorem~\ref{theorem1}, we have:
\begin{itemize}
\item[($H_6$)] $\Gamma \vdash \pc:= \pc + 1: \bot \; cmd$.
\end{itemize}
 

By the typing rule Assign-Counter, we have that:
\begin{itemize}
\item[($H_7$)] $\Gamma \vdash \pc: \bot \; Cvar$.
\end{itemize}



To prove ($G_1$), we rely on the Definition~\ref{def1} that states that $\mu_0 =_{\tau}^{\Gamma} \mu_1$, if $\forall y \in \mu_0$ such that $\Gamma(y) = \tau' \; var$ and $\tau' \leq \tau$ then $\mu_0(y) = \mu_1(y)$. For ($H_7$), $\Gamma(\pc) = \bot \; Cvar$. Since we are only interested in variables of type $\tau' \; var$, we conclude by ($H_5$) and ($H_1$) that $\mu'_0 =_{\tau}^{\Gamma} \mu'_1$.

To prove ($G_2$) we rely on the Definition~\ref{def2} and $(H_3)$. Since $t_0 = t_1 = \epsilon$ then $\Filter_{\tau}(t_0) = \Filter_{\tau}(t_1)$.

Since we proved ($G_1$) and ($G_2$), then $p$ is $NI^{\Gamma}_\tau$.

\subcase{$p \overset{\Delta}{=} \Sbc(L, e||\pc, K)$.}

Concerning ($G_1$) and ($G_2$), by Definition~\ref{def3}, we have that:
\begin{itemize}
\item[ ($H_1$)] $\mu_0 =_{\tau}^{\Gamma} \mu_1$.
\item[($H_2$)] $\mu_i \vdash \Sbc(L, e||\pc, K) \Rightarrow^{t_i} \mu_i$.
\end{itemize}

By the semantics rule Secure Broadcast, we have:
\begin{itemize}
\item[($H_3$)]  $t_i = \BEnc(L, vk_i, "m"||"v")$.
\item[($H_4$)]  $\mu_i \vdash K \Rightarrow v'_i$.
\item[($H_5$)] $\mu_i \vdash e \Rightarrow "m_i"$.
\item[($H_6$)] $\mu_i \vdash \pc \Rightarrow v_i$.
\item[($H_7$)] $\mu_i(L) = \{"n_0", "n_1", ..., "n_n" \}$.
\end{itemize}

By the hypothesis of Theorem~\ref{theorem1}, we have:
\begin{itemize}
\item[($H_8$)] $\Gamma \vdash \Sbc(L, e||\pc, K): \tau \; cmd$.
\end{itemize}

By ($H_8$) and the typing rule SBroadcast, we have:
\begin{itemize}
\item[($H_9$)] $\Gamma \vdash e: \tau$.
\item[($H_{10}$)] $\Gamma \vdash L: \tau \; Lvar$.
\item[($H_{11}$)] $\Gamma \vdash K: \top \; Kvar$.
\item[($H_{12}$)] $\Gamma \vdash \pc: \bot \; Cvar$.
\end{itemize}


To prove ($G_1$), we rely on the Definition~\ref{def1} that states that $\mu_0 =_{\tau}^{\Gamma} \mu_1$, if $\forall y \in \mu_0$ such that $\Gamma(y) = \tau' \; var$ and $\tau' \leq \tau$ then $\mu_0(y) = \mu_1(y)$. For ($H_9 - H_{12}$), all the variables types are different than $\tau' \; var$. Since we are only interested in variables of type $\tau' \; var$, ($G_1$) follows by ($H_1$) and ($H_2$).

To prove ($G_2$), we have two cases:
\begin{itemize}
\item[($H_{13}$)] $\Gamma(L) \leq \tau$

Being $\mu'_0 =_{\tau}^{\Gamma} \mu'_1$ and by ($H_2$) and($H_3$) we have that $t_0 = t_1$. For an execution that starts with $\mu_0$, $t_0 = \BEnc(L, vk_0, "m" || "v")$ and for an execution that starts with $\mu_1$, $t_1 = \BEnc(L, vk_1, "m" || "v")$. By ($H_{13}$), $\Gamma(L) \leq \tau$ then $\Filter(t_0) = [L, m]$ and $\Filter(t_1) = [L, m]$. It results that $\Filter(t_0) = \Filter(t_1)$ and therefore we prove $(G_2)$.

Since we proved ($G_1$) and ($G_2$), then $p$ is $NI^{\Gamma}_\tau$.

\item[($H_{14}$)] $\Gamma(L) \not\leq \tau$

Being $\mu'_0 =_{\tau}^{\Gamma} \mu'_1$ and by ($H_2$) and($H_3$) we have that $t_0 = t_1$. For an execution that starts with $\mu_0$, $t_0 = \BEnc(L, vk_0, "m" || "v")$ and for an execution that starts with $\mu_1$, $t_1 = \BEnc(L, vk_1, "m" || "v")$. By ($H_{14}$), $\Gamma(L) \not\leq \tau$, then  $\Filter_{\tau}(t_0) = [ \; ]$ and $\Filter_{\tau}(t_1) = [ \; ]$. It results that $\Filter_{\tau}(t_0) = \Filter_{\tau}(t_1)$ and therefore we prove $(G_2)$.

Since we proved ($G_1$) and ($G_2$), then $p$ is $NI^{\Gamma}_\tau$.
\end{itemize}

\subcase{$p \overset{\Delta}{=} \For ~ n \in L ~\Endorse ~ \Ra(L, L', n) $.}

Concerning ($G_1$) and ($G_2$), by Definition~\ref{def3}, we have that:
\begin{itemize}
\item[ ($H_1$)] $\mu_0 =_{\tau}^{\Gamma} \mu_1$.
\item[($H_2$)] $\mu_i \vdash \For ~ n \in L ~\Endorse ~ \Ra(L, L', n) \Rightarrow^{t_i} \mu'_i$.
\end{itemize}


By the semantics rule Endorse-Ra we have that:
\begin{itemize}
\item[($H_3$)] $t_i = \epsilon$.
\item[($H_4$)] $\mu_i \vdash \Ra(L) \Rightarrow S$.
\item[($H_5$)] $\mu_i(L) = \{"n_0", "n_1", ..., "n_n" \}$.
\item[($H_6$)] $\mu_i(L') = \{"n'_0", "n'_1", ..., "n'_n" \}$.
\end{itemize}

By ($H_2$) and the semantics rule Endorse-Ra we have that:
\begin{itemize}
\item[($H_7$)] $\mu'_i = \mu_i[L:= L \ S; L' := L' \cup S]$.
\end{itemize}


By the hypothesis of Theorem~\ref{theorem1}, we have:
\begin{itemize}
\item[($H_8$)] $\Gamma \vdash\For ~ n \in L ~\Endorse ~ \Ra(L, L', n): \tau \; cmd$.
\end{itemize}
 
By ($H_8$) and the typing rule Remote Attestation, we have that:
\begin{itemize}
\item[($H_9$)] $\Gamma \vdash L: \tau \; Lvar$.
\item[($H_{10}$)] $\Gamma \vdash L': \tau' \; Lvar$.
\item[($H_{11}$)] $\Gamma \vdash K: \top \; Kvar$.
\item[($H_{12}$)] $\Gamma \vdash \pc: \bot \; Cvar$.
\end{itemize}

To prove ($G_1$), we rely on the Definition~\ref{def1} that states that $\mu_0 =_{\tau}^{\Gamma} \mu_1$, if $\forall y \in \mu_0$ such that $\Gamma(y) = \tau' \; var$ and $\tau' \leq \tau$ then $\mu_0(y) = \mu_1(y)$. For ($H_9 - H_{12}$), all the variables types are different than $\tau' \; var$. Since we are only interested in variables of type $\tau' \; var$, ($G_1$) follows by ($H_1$) and ($H_2$).

To prove ($G_2$) we rely on the Definition~\ref{def2} and $(H_3)$. Since $t_0 = t_1 = \epsilon$ then $\Filter_{\tau}(t_0) = \Filter_{\tau}(t_1)$.

Since we proved ($G_1$) and ($G_2$), then $p$ is $NI^{\Gamma}_\tau$.

\case{Case: height $\leq n$}
We will state our inductive hypothesis for a program c:

For the hypothesis of the theorem, we have that:
\begin{itemize}
\item[($IH_1$)] $\Gamma \vdash c: \tau ~ cmd$.
\item[($IH_2$)] $\mu_0 =_{\tau}^{\Gamma} \mu_1$.
\end{itemize}

For the derivation tree of height $\leq n$, we have that:
\begin{itemize}
\item[($IH_3$)] $\mu_0 \vdash c \Rightarrow^{t_0} \mu'_0$.
\item[($IH_4$)] $\mu_1 \vdash c \Rightarrow^{t_1} \mu'_1$.
\end{itemize}

Then we conclude that:
\begin{itemize}
\item[($IH_5$)] $\mu'_0 =_{\tau}^{\Gamma} \mu'_1$.
\item[($IH_6$)] $\Filter_{\tau}(t_0) = \Filter_{\tau}(t_1)$.
\end{itemize}

We suppose that ($IH_1 - IH_6$) are valid for programs of  height $\leq n$ of typing derivation tree $\Gamma \vdash p$.


\case{Inductive case: height$=n+1$}
\subcase{$p \overset{\Delta}{=} c'; c''$.}

We want to prove that $\Gamma \vdash c'; c''$ is $NI_{\tau}^{\Gamma}$. We assume that $c'$ and $c''$ are of height $\leq n$, and $c'; c''$ of height $n+1$.

For the typing rule Sequence, we have that:

\begin{itemize}
\item[($H_1$)] $\Gamma \vdash c': \tau ~ cmd$.
\item[($H_2$)] $\Gamma \vdash c'': \tau ~ cmd$.
\end{itemize}

For the semantics rule Sequence, we have that:

\begin{itemize}
\item[($H_3$)] $\mu_i \vdash c' \Rightarrow^{t'_i} \mu'_i$.
\item[($H_4$)] $\mu'_i \vdash c'' \Rightarrow^{t''_i} \mu''_i$.
\end{itemize}


Concerning $c'$, from the inductive hypotheses, it follows that:
\begin{itemize}
\item[($H_5$)]  $\mu_{c'_0} =_{\tau}^{\Gamma} \mu_{c'_1}$.
\item[($H_6$)] $\mu_{c'_0} \vdash c \Rightarrow^{t'_0} \mu'_{c'_0}$  
\item[($H_7$)] $\mu_{c'_1} \vdash c \Rightarrow^{t'_1} \mu'_{c'_1}$
\item[($H_8$)] $\mu'_{c'_0} =_{\tau}^{\Gamma} \mu'_{c'_1}$.
\item[($H_9$)] $\Filter_{\tau}(t'_0) = \Filter_{\tau}(t'_1)$.
\end{itemize}

It follows that $c'$ is $NI_{\tau}^{\Gamma}$.


Concerning $c''$, from the inductive hypotheses, it follows that:
\begin{itemize}
\item[($H_{10}$)]  $\mu_{c''_0} =_{\tau}^{\Gamma} \mu_{c''_1}$.
\item[($H_{11}$)] $\mu_{c''_0} \vdash c \Rightarrow^{t''_0} \mu'_{c''_0}$  
\item[($H_{12}$)] $\mu_{c''_1} \vdash c \Rightarrow^{t''_1} \mu'_{c''_1}$
\item[($H_{13}$)] $\mu'_{c''_0} =_{\tau}^{\Gamma} \mu'_{c''_1}$.
\item[($H_{14}$)] $\Filter_{\tau}(t''_0) = \Filter_{\tau}(t''_1)$.
\end{itemize}

It follows that $c''$ is $NI_{\tau}^{\Gamma}$.

In the semantics rule Sequence, we use $t' \cdot t''$ to denote the concatenation of two traces $t'$ and $t''$. By ($H_9$) and ($H_{14}$) we have that $t'_0 = t'_1$ and $t''_0 = t''_1$, which implies that $t'_0 \cdot t''_0 = t'_1 \cdot t''_1$. We conclude that  $\Filter_{\tau}(t'_0 \cdot t''_0) = \Filter_{\tau}(t'_1 \cdot t''_1)$.

Since $c'$, $c''$ are $NI_{\tau}^{\Gamma}$ and $\Filter_{\tau}(t'_0 \cdot t''_0) = \Filter_{\tau}(t'_1 \cdot t''_1)$, we conclude that $\Gamma \vdash c'; c''$ is $NI_{\tau}^{\Gamma}$.

\subcase{$p \overset{\Delta}{=} \While ~e ~ \Do ~ c'$.}

For the typing rule While, we have that:

\begin{itemize}
\item[($H_1$)] $\Gamma \vdash c': \tau' ~ cmd$.
\item[($H_2$)] $\Gamma \vdash e: \tau'$.
\end{itemize}


In the following, we show by induction that $c'$ is $NI_{\tau}^{\Gamma}$.

By $(H_1)$ and because the height of $\Gamma \vdash \While ~ e ~ \Do ~ c'$ is $n+1$, we know that the height of the typing derivation tree of $(H_1)$ is $\leq n$. Hence, we can apply the inductive hypothesis and get:

\begin{itemize}
\item[($H_3$)] $\mu_0 =_{\tau}^{\Gamma} \mu_1$.
\item[($H_4$)] $\mu_0 \vdash c' \Rightarrow^{t_0} \mu'_0$.
\item[($H_5$)] $\mu_1 \vdash c' \Rightarrow^{t_1} \mu'_1$.
\end{itemize}

Then, 
\begin{itemize}
\item[($H_6$)] $\mu'_0 =_{\tau}^{\Gamma} \mu'_1$.
\item[($H_7$)] $\Filter_{\tau}(t_0) = \Filter_{\tau}(t_1)$.
\end{itemize}


We want to prove that $\While ~ e ~ \Do ~ c'$ is $NI_{\tau}^{\Gamma}$. By the hypothesis of the theorem, we have that if:
\begin{itemize}
\item[($H'_0$)] $\Gamma \vdash \While ~ e ~ \Do ~ c': \tau' ~ cmd$.
\item[($H'_1$)] $\mu_0 =_{\tau}^{\Gamma} \mu_1$.
\item[($H'_2$)] $\mu_0 \vdash \While ~ e ~ \Do ~ c \Rightarrow^{t_0} \mu'_0$.
\item[($H'_3$)] $\mu_1 \vdash \While ~ e ~ \Do ~ c \Rightarrow^{t_1} \mu'_1$.
\end{itemize}

Then, we want to prove that:

\begin{itemize}
\item[($G_0$)] $\mu'_0 =_{\tau}^{\Gamma} \mu'_1$.
\item[($G_1$)] $\Filter_{\tau}(t_0) = \Filter_{\tau}(t_1)$.
\end{itemize}

By ($H'_2$) and the semantics rule of Loop, we have that:
\begin{itemize}
\item[($H'_4$)] $\mu_0 \vdash e \Rightarrow v_0$.
\end{itemize}

By ($H'_3$) and the semantics rule of Loop, we have that:
\begin{itemize}
\item[($H'_5$)] $\mu_1 \vdash e \Rightarrow v_1$.
\end{itemize}

Depending on $\tau'$, we have two cases:
\subsubcase{$\Gamma \vdash e: \tau',~ \tau' \leq \tau$}.
\begin{itemize}
\item[($H'_6$)] $\tau' \leq \tau$.
\end{itemize}

By Lemma~\ref{lemma1} that can be applied on ($H'_1$), ($H_2$), ($H'_6$), ($H'_4$) and ($H'_5$), we conclude that $v_0 = v_1$.


We prove this case by induction on the height of the semantics tree of ($H'_2$). (We do not show this formally, but we rely on the fact that the height of ($H'_2$) is equal to the height of ($H'_3$) by Lemma~\ref{lemma1}).

\paragraph{Base case: height = 2.}

The only possibility for the semantics tree to be of height $=2$ is:
\begin{itemize}
\item $v_0 = v_1 = False$.
\end{itemize}

\begin{mathpar}
\small
\inferrule*
{\mu_0 \vdash e \Rightarrow False}
{\mu_0 \vdash \While ~ e ~ \Do ~ c \Rightarrow^{t_0}  \mu_0 }
\end{mathpar}

\begin{mathpar}
\small
\inferrule*
{\mu_1 \vdash e \Rightarrow False}
{\mu_1 \vdash \While ~ e ~ \Do ~ c \Rightarrow^{t_1}  \mu_1 }
\end{mathpar}


We have that $\mu'_0 = \mu_0$ and  $\mu'_1 = \mu_1$ . We conclude by ($H'_1$) that $\mu'_0 =_{\tau}^{\Gamma} \mu'_1$. Moreover, since $t_0 = t_1 = \epsilon$ then $\Filter_{\tau}(t_0) = \Filter_{\tau}(t_1)$.

Assuming that our inductive hypothesis holds for the case of While when evaluating the height of semantics tree $\leq m$ with $\Gamma \vdash e: \tau', \tau' \leq \tau$, let us prove the case of While with height $= m+1$.

\paragraph{Inductive case: height = m+1.}

\begin{mathpar}
\small
\inferrule*
{\mu_0 \vdash e \Rightarrow True \\ \mu_0 \vdash c' \Rightarrow^{t_0}  \mu''_0 \\ (H'_7) ~ \mu''_0 \vdash \While ~ e ~ \Do ~ c' \Rightarrow^{t'_0}  \mu'_0 }
{\mu_0 \vdash \While ~ e ~ \Do ~ c' \Rightarrow^{t_0 \cdot t'_0}  \mu'_0}
\end{mathpar}

The height of ($H'_2$) is $m+1$.

\begin{mathpar}
\small
\inferrule*
{\mu_1 \vdash e \Rightarrow True \\ \mu_1 \vdash c' \Rightarrow^{t_1}  \mu''_1 \\ (H'_8) ~ \mu''_1 \vdash \While ~ e ~ \Do ~ c' \Rightarrow^{t'_1}  \mu'_1 }
{\mu_1 \vdash \While ~ e ~ \Do ~ c' \Rightarrow^{t_1 \cdot t'_1}  \mu'_1}
\end{mathpar}

The height of ($H'_3$) is $m+1$.


By the previous induction on $c'$, we have that $\mu''_0 =_{\tau}^{\Gamma} \mu''_1$ and $\Filter_{\tau}(t_0) = \Filter_{\tau}(t_1)$, through $\mu_0 =_{\tau}^{\Gamma} \mu_1$, $\mu_0 \vdash c' \Rightarrow^{t_0} \mu''_0$ and $\mu_1 \vdash c' \Rightarrow^{t_1} \mu''_1$.

Moreover, by induction on $\While ~ e ~ \Do ~ c'$ by $(H'_7)$ and $(H'_8)$, we can conclude that $\mu'_0 =_{\tau}^{\Gamma} \mu'_1$ which is already our goal $(G_1)$ and $\Filter_{\tau}(t'_0) = \Filter_{\tau}(t'_1)$. This is because we have that  $\mu''_0 =_{\tau}^{\Gamma} \mu''_1$ and  $\mu''_0 \vdash \While ~ e ~ \Do ~ c' \Rightarrow^{t'_0}  \mu'_0$ and $\mu''_1 \vdash \While ~ e ~ \Do ~ c' \Rightarrow^{t'_1}  \mu'_1$.

Since $t_i \cdot t'_i$ is the concatenation of two traces $t_i$ and $t'_i$, and since $\Filter_{\tau}(t_0) = \Filter_{\tau}(t_1)$ and $\Filter_{\tau}(t'_0) = \Filter_{\tau}(t'_1)$, we conclude that $\Filter_{\tau}(t_0 \cdot t'_0) = \Filter_{\tau}(t_1 \cdot t'_1)$ from which $(G_2)$ follows.

Since $(G_1)$ and $(G_2)$ are satisfied, then $\While ~ e ~ \Do ~ c'$ is $NI_{\tau}^{\Gamma}$.



\subsubcase{$\Gamma \vdash e: \tau',~ \tau' \not\leq \tau$}.



\begin{lemma}(HighCommand)\label{lemma2}
$\forall \tau, \tau', \mu_i$, if $\Gamma \vdash c: \tau' ~ cmd$ and $\tau' \not\leq \tau$ and $\mu_i \vdash c \Rightarrow \mu'_i$, then $\mu_i =_{\tau}^{\Gamma} \mu'_i$
\end{lemma}
























%
%By induction, we know that $c'$ is $NI_{\tau}^{\Gamma}$ by relying on $(IH_5)$ and $(IH_6)$. In what follows, we need to prove that $\While ~ e ~ \Do ~ c'$ is $NI_{\tau}^{\Gamma}$.
%
%For the semantics rule Loop, we have that:
%
%\begin{itemize}
%\item[($H_3$)] $\mu_i \vdash e \Rightarrow v_i$.
%%\item[($H_3'$)] $t_i = \epsilon$.
%%\item[($H_4$)] $\mu_i \vdash c' \Rightarrow^{t_i} \mu'_i$.
%%\item[($H_4'$)] $\mu_i \vdash c' \Rightarrow^{t_i} \mu_i$.
%%\item[($H_5$)] $\mu'_i \vdash \While ~ e ~ \Do ~ c' \Rightarrow^{t'_i} \mu''_i$.
%%
%%\end{itemize}
%
%By ($H_1$) and because the height of $\Gamma \vdash p$ is $n+1$, we know that the height of the typing derivation tree of ($H_1$) is $\leq n$. Hence, we can apply the inductive hypothesis and get:
%\begin{itemize}
%\item[($H_4$)] $\mu_0 =_{\tau}^{\Gamma} \mu_1$.
%\item[($H_5$)] $\mu_0 \vdash c' \Rightarrow^{t_0} \mu'_0$.
%\item[($H_6$)] $\mu_1 \vdash c' \Rightarrow^{t_1} \mu'_1$.
%\item[($H_7$)] $\mu'_0 =_{\tau}^{\Gamma} \mu'_1$.
%\item[($H_8$)] $\Filter_{\tau}(t_0) = \Filter_{\tau}(t_1)$.
%\end{itemize}
%
%Depending on $\tau'$, we have two cases:
%\subsubcase{$\Gamma \vdash e: \tau',~ \tau' \leq \tau$}.
%\begin{itemize}
%\item[($H_9$)] $\tau' \leq \tau$.
%\end{itemize}
%
%
%By Lemma~\ref{lemma1} that can be applied due to ($H_4$), ($H_9$), ($H_2$) and ($H_3$), then 
%\begin{itemize}
%\item[($H_{10}$)] $v_0 = v_1$.
%\end{itemize}
%
%\paragraph{Base case: heightS = 2.}
%
%\begin{mathpar}
%\small
%\inferrule*
%{\mu_0 \vdash e \Rightarrow False}
%{\mu_0 \vdash \While ~ e ~ \Do ~ c \Rightarrow^{t_0}  \mu_0 }
%\end{mathpar}
%
%\begin{mathpar}
%\small
%\inferrule*
%{\mu_1 \vdash e \Rightarrow False}
%{\mu_1 \vdash \While ~ e ~ \Do ~ c \Rightarrow^{t_1}  \mu_1 }
%\end{mathpar}
%
%
%\begin{itemize}
%\item $v_0 = v_1 = False$.
%\end{itemize}
%
%
%From the semantics rule of Loop-False, we have that:
%\begin{itemize}
%\item[($H_{11}$)] $t_i = \epsilon$.
%\end{itemize}
%
%For ($H_{10}$) and  ($H_4$) , we conclude that $\mu_0 =_{\tau}^{\Gamma} \mu_1$. By ($H_{11}$) we conclude that $\Filter_{\tau}(t_0) = \Filter_{\tau}(t_1)$, since $t_0 = t_1 = \epsilon$.
%
%
%
%
%\begin{itemize}
%\item $v_0 = v_1 = True$.
%\end{itemize}
%We prove this case by induction on the number of iterations of While, which is equivalent to prove the induction height semantics of the derivation tree of While.
%
%
%\paragraph{Base case: heightS = 3.}
%
%\begin{mathpar}
%\small
%\inferrule*
%{\mu_0 \vdash e \Rightarrow True \\ \mu_o \vdash c \Rightarrow^{t_0} \mu'_0 \\ \inferrule*{\mu'_0 \vdash e \Rightarrow False }{\mu'_0 \vdash \While ~ e ~ \Do ~ c \Rightarrow^{t_0}  \mu'_0 }}
%{\mu_0 \vdash \While ~ e ~ \Do ~ c \Rightarrow^{t_0}  \mu'_0 }
%\end{mathpar}
%
%\begin{mathpar}
%\small
%\inferrule*
%{\mu_1 \vdash e \Rightarrow True \\ \mu_1 \vdash c \Rightarrow^{t_1}  \mu'_1\\ \inferrule*{\mu'_1 \vdash e \Rightarrow False }{\mu'_1 \vdash \While ~ e ~ \Do ~ c \Rightarrow^{t_1}  \mu'_1 }}
%{\mu_1 \vdash \While ~ e ~ \Do ~ c \Rightarrow^{t_1}  \mu'_1 }
%\end{mathpar}
%
%
%
%Using the hypothesis of the theorem, $\Gamma \vdash \While ~ e ~ \Do ~ c': \tau' ~ cmd$ and $\mu_0 =_{\tau}^{\Gamma} \mu_1$, we state the inductive hypotheses:
%\begin{itemize}
%\item[($IHS_1$)] $\Gamma \vdash \While ~ e ~ \Do ~ c': \tau' ~ cmd$.
%\item[($IHS_2$)] $\mu_0 =_{\tau}^{\Gamma} \mu_1$.
%\item[($IHS_3$)] $\mu_0 \vdash \While ~ e ~ \Do ~ c \Rightarrow^{t_0} \mu'_0$.
%\item[($IHS_4$)] $\mu_1 \vdash \While ~ e ~ \Do ~ c \Rightarrow^{t_1} \mu'_1$.
%\end{itemize}
%Then we conclude that:
%\begin{itemize}
%\item[($IHS_5$)] $\mu'_0 =_{\tau}^{\Gamma} \mu'_1$.
%\item[($IHS_6$)] $\Filter_{\tau}(t_0) = \Filter_{\tau}(t_1)$.
%\end{itemize}

%We suppose that ($IHS_1$ - $IHS_5$) are valid for programs of heightS $\leq$ n of typing derivation tree $\Gamma \vdash p$.
%For ($H_{10}$) and by the induction hypotheses, we can conclude that $\mu'_0 =_{\tau}^{\Gamma} \mu'_1$ and $\Filter_{\tau}(t_0) = \Filter_{\tau}(t_1)$.

%\paragraph{HeightS $\leq$ n.}
%
%Using the hypothesis of the theorem, $\Gamma \vdash \While ~ e ~ \Do ~ c': \tau' ~ cmd$ and $\mu_0 =_{\tau}^{\Gamma} \mu_1$,
%\begin{itemize}
%\item[($IHS_1$)] $\Gamma \vdash \While ~ e ~ \Do ~ c': \tau' ~ cmd$.
%\end{itemize}
%
%From ($IHS_1$), it follows that:
%\begin{itemize}
%\item[($IHS_2$)] $\mu_0 =_{\tau}^{\Gamma} \mu_1$.
%\item[($IHS_3$)] $\mu_i \vdash \While ~ e ~ \Do ~ c \Rightarrow^{t_i} \mu'_i$.
%\item[($IHS_4$)] $\mu'_0 =_{\tau}^{\Gamma} \mu'_1$.
%\item[($IHS_5$)] $\Filter_{\tau}(t_0) = \Filter_{\tau}(t_1)$.
%\end{itemize}
%
%We suppose that ($IHS_1$ - $IHS_5$) are valid for programs of heightS $\leq$ n of typing derivation tree $\Gamma \vdash p$.
%
%By ($IHS_1$), it follows that:
%\begin{itemize}
%\item[($H_{12}$)]  $\mu_0 =_{\tau}^{\Gamma} \mu_1$.
%\item[($H_{13}$)]  $\mu_0 \vdash \While ~ e ~ \Do ~ c \Rightarrow^{t_0} \mu'_0$.
%\item[($H_{14}$)]  $\mu_1 \vdash \While ~ e ~ \Do ~ c \Rightarrow^{t_1} \mu'_1$.
%\end{itemize}
%
%We conclude that:
%\begin{itemize}
%\item[($H_{15}$)]  $\mu'_0 =_{\tau}^{\Gamma} \mu'_1$.
%\item[($H_{16}$)] $\Filter_{\tau}(t'_0) = \Filter_{\tau}(t'_1)$.
%\end{itemize}
%Concerning $\mu_i \vdash \While ~ e ~ \Do ~ c \Rightarrow^{t_i} \mu'_i$, by ($IHS_1$) it follows that:
%\begin{itemize}
%\item[($H_6$)] $\mu'_0 =_{\tau}^{\Gamma} \mu'_1$.
%\item[($H_7$)] $\Filter_{\tau}(t_0) = \Filter_{\tau}(t_1)$.
%\end{itemize}
%\paragraph{Inductive case: heightS = $n+1$}.
%
%We want to prove that $\While ~ e ~ \Do ~ c$ is $NI_{\tau}^{\Gamma}$.
%
%
%
%\begin{mathpar}
%\small
%\inferrule*
%{\mu_0 \vdash e \Rightarrow True \\ \mu_0 \vdash c \Rightarrow^{t_0} \mu'_0 \\ \inferrule*{(height = n)}{\mu'_0 \vdash \While ~ e ~ \Do ~ c \Rightarrow^{t'_0}  \mu''_0}}
%{\mu_0 \vdash \While ~ e ~ \Do ~ c \Rightarrow^{t_0 \cdot t'_0}  \mu''_0 }
%\end{mathpar}
%
%\begin{mathpar}
%\small
%\inferrule*
%{\mu_1 \vdash e \Rightarrow True \\ \mu_1 \vdash c \Rightarrow^{t_1} \mu'_1 \\ \inferrule*{(height = n)}{\mu'_1 \vdash \While ~ e ~ \Do ~ c \Rightarrow^{t'_1}  \mu''_1}}
%{\mu_1 \vdash \While ~ e ~ \Do ~ c \Rightarrow^{t_1 \cdot t'_1}  \mu''_1 }
%\end{mathpar}



%By ($H_5$) have that:
%\begin{itemize}
%\item[($H_5'$)] $\mu'_0 \vdash \While ~ e ~ \Do ~ c' \Rightarrow^{t'_0} \mu''_0$.
%\item[($H_5''$)] $\mu'_1 \vdash \While ~ e ~ \Do ~ c' \Rightarrow^{t'_1} \mu''_1$.
%\end{itemize}
%
%The height of $\mu'_0 \vdash \While ~ e ~ \Do ~ c' \Rightarrow^{t'_0} \mu''_0$ and $\mu'_1 \vdash \While ~ e ~ \Do ~ c' \Rightarrow^{t'_1} \mu''_1$ is $n$. 
%
%Since, $\mu_0 =_{\tau}^{\Gamma} \mu_1$ by ($H_{12}$) and $\mu'_0 =_{\tau}^{\Gamma} \mu'_1$ by ($H_{15}$) and $\mu'_0 \vdash \While ~ e ~ \Do ~ c' \Rightarrow^{t'_0} \mu''_0$ by ($H_5'$) and $\mu'_1 \vdash \While ~ e ~ \Do ~ c' \Rightarrow^{t'_1} \mu''_1$ by  ($H_5''$), we conclude that $\mu''_0 =_{\tau}^{\Gamma} \mu''_1$.
%
%%\begin{itemize}
%\item[($H_8$)] $\mu''_0 =_{\tau}^{\Gamma} \mu''_1$.
%\item[($H_9$)] $\Filter_{\tau}(t'_0) = \Filter_{\tau}(t'_1)$.
%\end{itemize}


%We write $t \cdot t'$ to denote the concatenation of two traces $t$ and $t'$.  By ($H_7$) and ($H_9$) we have that $t_0 = t_1$ and $t'_0 = t'_1$, which implies that $t_0 \cdot t'_0 = t_1 \cdot t'_1$. We conclude that  $\Filter_{\tau}(t_0 \cdot t'_0) = \Filter_{\tau}(t_1 \cdot t'_1)$.
%
%Since $e, c, \While ~ e ~ \Do ~ c$ are $NI_{\tau}^{\Gamma}$, we conclude that p is $NI_{\tau}^{\Gamma}$.
%\end{itemize}


%Concerning $\Gamma \vdash e$, to prove ($G_1$), we rely on the Definition~\ref{def1} that states that $\mu_0 =_{\tau}^{\Gamma} \mu_1$, if $\forall y \in \mu_0$ such that $\Gamma(y) = \tau' \; var$ and $\tau' \leq \tau$ then $\mu_0(y) = \mu_1(y)$. Since the type of $e$ is $\tau'$ and we are only interested in variables of type $\tau' \; var$, ($G_1$) follows by ($H_3$) and ($H_6$).
%
%To prove ($G_2$) we rely on the Definition~\ref{def2} and $(H_3')$. Since $t_0 = t_1 = \epsilon$ then $\Filter_{\tau}(t_0) = \Filter_{\tau}(t_1)$.
%
%Since $e, c, \While ~ e ~ \Do ~ c$ are $NI_{\tau}^{\Gamma}$, we conclude that p is $NI_{\tau}^{\Gamma}$.










%
%\subsubcase{$\Gamma \vdash e: \tau', \tau' \not\leq \tau$}.
%\begin{itemize}
%\item[($H_{12}$)] $\tau' \not\leq \tau$.
%\end{itemize}
%\begin{lemma}(HighExpression)\label{lemma2}
%$\forall \tau, \tau', \mu_i$, if $\Gamma \vdash c: \tau' ~ cmd$ and $\tau' \not\leq \tau$ and $\mu_i \vdash c \Rightarrow \mu'_i$, then $\mu_i =_{\tau}^{\Gamma} \mu'_i$
%\end{lemma}



%\subcase{$p \overset{\Delta}{=} \If ~ e ~ \Then c' ~ \Else ~ c''$}.
%
%For the typing rule If, we have that:
%\begin{itemize}
%\item[$(H_1)$] $\Gamma \vdash e: \tau'$.
%\item[$(H_2)$] $\Gamma \vdash c': \tau' ~ cmd$.
%\item[$(H_3)$] $\Gamma \vdash c'': \tau' ~ cmd$.
%\end{itemize}
%
%For the semantics rule Branch, we have that:
%\begin{itemize}
%\item[$(H_4)$] $\mu_i \vdash e \Rightarrow v_i$.
%\item[$(H_5)$] $\mu_i \vdash c' \Rightarrow_{t_i} \mu'_i$.
%\item[$(H_6)$] $\mu_i \vdash c'' \Rightarrow_{t_i} \mu'_i$.
%\end{itemize}
%
%Concerning $c'$, by induction, we conclude that:
%\begin{itemize}
%\item[$(H_7)$] $\mu'_{c'_0} =_{\tau}^{\Gamma} \mu'_{c'_1}$.
%\item[$(H_8)$] $\Filter_{\tau}(t_0) = \Filter_{\tau}(t_1)$.
%\end{itemize}
%
%It follows that $c'$ is $NI_{\tau}^{\Gamma}$.
%
%Concerning $c''$, by induction, we conclude that:
%\begin{itemize}
%\item[$(H_7)$] $\mu'_{c''_0} =_{\tau}^{\Gamma} \mu'_{c''_1}$.
%\item[$(H_8)$] $\Filter_{\tau}(t_0) = \Filter_{\tau}(t_1)$.
%\end{itemize}
%
%It follows that $c''$ is $NI_{\tau}^{\Gamma}$.
%
%Depending on $\tau'$, we have two cases:
%
%\subsubcase{$\Gamma \vdash e: \tau', \tau' \leq \tau$}.
%\begin{itemize}
%\item[$(H_9)$] $\tau' \leq \tau$.
%\end{itemize}
%By Lemma~\ref{lemma1}, we can conclude that:
%\begin{itemize}
%\item[$(H_{10})$]  $v_0 = v_1$.
%\end{itemize}
%
%
%To prove ($G_1$), we rely on the Definition~\ref{def1} that states that $\mu_0 =_{\tau}^{\Gamma} \mu_1$, if $\forall x \in \mu_0$ such that $\Gamma(x) = \tau' \; var$ and $\tau' \leq \tau$ then $\mu_0(x) = \mu_1(x)$. Since we are only interested by variables of type $\tau' ~ var$, we conclude that $\mu'_{0} =_{\tau}^{\Gamma} \mu'_{1}$.
%
%Filters?
%
%
%
%
%
%\subsubcase{$\Gamma \vdash e: \tau', \tau' \not\leq \tau$}.
%
%\begin{itemize}
%\item[$(H_{11})$] $\tau' \not\leq \tau$.
%\end{itemize}
%
%To prove ($G_1$), we rely on the Definition~\ref{def1} that states that $\mu_0 =_{\tau}^{\Gamma} \mu_1$, if $\forall x \in \mu_0$ such that $\Gamma(x) = \tau' \; var$ and $\tau' \leq \tau$ then $\mu_0(x) = \mu_1(x)$. For $(H_{11})$ we have that $\tau' \not\leq \tau$. Since we are only interested by expressions with security level less or equal than $\tau$, we conclude that $\mu'_{0} =_{\tau}^{\Gamma} \mu'_{1}$.
%
%Filters?

\end{proof}
\end{document}
